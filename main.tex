\documentclass[11pt]{article}
    \title{\textbf{Bash: команды, фишки и примеры}}
    \author{Someanonimcoder}
    \date{}
    
    \addtolength{\topmargin}{-3cm}
    \addtolength{\textheight}{3cm}
\usepackage{listings}

\usepackage[utf8]{inputenc}
\usepackage[russian]{babel}


\begin{document}

\maketitle
\thispagestyle{empty}

\section*{Intro}
В этом документе я хочу собрать основные команды bash и объяснить их для тех, кто работает на mac или linux

Любые дополнения, исправления и уточнения приветствуются.

\section*{Basic}
\subsection*{Особое}
Некоторые "магические" папки:
\begin{lstlisting}[language=bash]
   ~    # home dir 
   ..   # one dir up
   -    # one dir back
   .    # current dir
\end{lstlisting}
Перенаправление вывода команды:
\begin{lstlisting}[language=bash]
  # this is a comment. bash doesn't read it
  # $ is a prompt for command input
  # print output not in terminal but in file.txt
  $ command > file.txt     
  # send output of command1 to command2 as (n+1)th argument
  $ command1 |command2 arg1 arg2 ... argn
\end{lstlisting}
Читай мануалы, \$\&\#\%*!
\begin{lstlisting}[language=bash]
  # print help for command
  $ man command     
\end{lstlisting}
Как красиво попросить помощи у товарища? Читать километр вывода в вк или тг ну очень неудобно. Поэтому пожалуйста, используйте $pastein.it$. Это сайт, где ваш код или вывод терминала красиво оформится. Пример: $pastebin.com/hLw6DhQm$.

Еще есть команда pastebinit, она требует установки, но позволяет делать так:
\begin{lstlisting}[language=bash]
  $ command
  <A wall of strange errors>
  $ command |pastebinit
  pastebin.com/AbCdEfGh
\end{lstlisting}
Потом можно скопировать ссылку и отправить мне, ИВС, ММ, кому угодно еще. Так мы быстрее прочитает Эту вашу стену сообщений об ошибках и поможем. 

ОСТОРОЖНО!!! Все тексты публичны, так что пароли, ключи, персональные данные... ну вы поняли.


\subsection*{Навигация}
Сюда я собрал команды для поиска и перемещения файлов, перехода по папкам, и всего такого
\subsubsection*{cd}
Переход в папку выполняется командой 
\begin{lstlisting}[language=bash]
  $ cd <name>
\end{lstlisting}

\subsubsection*{cp}
Скопировать 
\begin{lstlisting}[language=bash]
  # copy file to dir
  $ cp file dir
  #copy file to dir as file2
  $ cp file dir/file2
\end{lstlisting}

\subsubsection*{mv}
Переместить(move)

Идентично cp по логике работы

\subsubsection*{rm}
Удаляет файл. ВНИМАНИЕ!!! Удаляет не в корзину, а по-взрослому. Без $-rf$ не удаялет папки
\begin{lstlisting}[language=bash]
  #remove file
  $ rm file
  #remove directory
  $ rm -rf dir
\end{lstlisting}

\subsubsection*{ls}
Список файлов и папок можно просмотреть командой:
\begin{lstlisting}[language=bash]
  $ ls <name>
\end{lstlisting}
<name> здесь - имя той папки, список файлов в которой вы хотите увидеть



\subsubsection*{file}
Эта команда выдает информацию о файле:
\begin{lstlisting}[language=bash]
  $ file MeCl.inp 
MeCl.inp: ASCII text, with CRLF line terminators
  $ file DZ1.zip 
DZ1.zip: Zip archive data, at least v2.0 to extract
  $ file doc.docx 
doc.docx: Microsoft Word 2007+
\end{lstlisting}

\subsubsection*{find}

Чтобы найти файл из терминала, используйте эту команду. Немного примеров:
\begin{lstlisting}[language=bash]
  $ find .                    
  #show all files in current dir
  $ find Docs -name "*.pdf"   
  #show files by name "*.pdf" in "Docs"
  $ find Docs -name "*.pdf"   
  #same without register(file=fIlE)
  $find Docs -not -name "*.pdf"   
  #same, but NOT "*.pdf" files
\end{lstlisting}


\subsection*{Работа с текстом}
Команды для эффективного просмотра, редактирования, анализа текста из командной строки
\subsubsection*{less}
Показывает файл. Выход - $q$. Навигация - стрелочки и $pg up$/$pg dn$. Больше сказать и нечего.

\subsubsection*{more}
Более примитивный вариант less, что иронично с учетом их названий
\subsubsection*{cat}
Печатает файл в терминал. Пример, выводящий код программы лежащий в файле $rk.c$. Вывод сокращен
\begin{lstlisting}[language=bash]
$ cat rk.c
#ifdef _OPENMP
#include <omp.h>
#endif

#include <iostream>

<..5000 lines..>

  std::cout << result << std::endl;
}
\end{lstlisting}

Не всегда удобно, особенно если файл на пару тысяч строк.
\subsubsection*{head}
Печатает <<голову>>- 10 первых строк
\begin{lstlisting}[language=bash]
 $ head rk.c
#ifdef _OPENMP
#include <omp.h>
#endif

#include <iostream>

constexpr int size = 1000000;

inline double f(double x, double y) { return x * x; }
\end{lstlisting}

\subsubsection*{tail}
Печатает <<хвост>> - 10 последних строк
\begin{lstlisting}[language=bash]
 $ tail rk.c
  for (int i = 0; i < threads; i++) {
    std::cout << "Thread " << i << " " << 0 + (double)size / (double)threads * i
              << " " << (double)size / (double)threads * (i + 1) << "\n";
    result +=
        solveoderungekutta((double)size / (double)threads * i,
                           (double)size / (double)threads * (i + 1), 0, 0.0001);
  }

  std::cout << result << std::endl;
}
\end{lstlisting}

\subsubsection*{grep}

Ищет заданный паттерн(первый аргумент) внутри второго аргумента. В примере ищет "energy" в выводе первой команды. Именно так grep чаще всего и применяют.

\begin{lstlisting}
#second argument is output of previous command
$ cat NH3.out |grep "energy" 
Change in XC energy                          ...     0.000026139
Gradient of the Kohn-Sham DFT energy:
<...>
Change in XC energy                          ...     0.000025435

\end{lstlisting} 

\subsubsection{sed}
Мощный потоковый редактор файлов. Принимает два аргумента - строку-"шаблон" и то что хотим отредактировать
\begin{lstlisting}
 #s=substitute
 $ echo "hello world" |sed "s/hello/goodbye/"
goodbye world
 #only first!
 $ echo "hello world" |sed "s/l/6/"
he6lo world
 #g=global, all occurences
 $ echo "hello world" |sed "s/l/6/g"
he66o wor6d
\end{lstlisting}

\subsubsection*{nano}
Няшный редактор в командной строке. Простой и интуитивно понятный.
 \begin{lstlisting}[language=bash]
 $ nano
 GNU nano 4.8 New Buffer


Text you are editing is here

At the end there is a manul for nano - nano shows it!



[ line 1/1 (100%), col 1/1 (100%), char 0/0 (0%) ]
^G Get Help     ^O Write Out    ^W Where Is     ^K Cut Text     ^J Justify      ^C Cur Pos      M-U Undo       
^X Exit         ^R Read File    ^\ Replace      ^U Paste Text   ^T To Spell     ^_ Go To Line   M-E Redo       
\end{lstlisting}

\subsubsection*{vi(vim)}
Стращный редактор. Очень удобный отдельным гражданам, но сложный для начала работы. Если зашли случайно - <esc>, потом наберите $:q!$ и забудьте как страшный сон.

\subsection*{Системное}
Здесь собраны команды для получения информации о системе и взаимодействия с ней.

\subsubsection*{top}
Показывает процессы и загрузку машины. Некрасивый но вроде стоит везде "из коробки"
\begin{lstlisting}[language=bash]
Tasks: 263 total,   1 running, 262 sleeping,   0 stopped,   0 zombie        
%Cpu(s):  1,4 us,  1,4 sy,  0,0 ni, 96,4 id,  0,0 wa,  0,0 hi,  0,7 si,  0  
MiB Mem :   3821,4 total,    450,7 free,   1937,2 used,   1433,6 buff/cach  
MiB Swap:   8192,0 total,   7674,7 free,    517,3 used.   1403,6 avail Mem  
                                                                            
    PID USER      PR  NI    VIRT    RES    SHR S  %CPU  %MEM     TIME+      
 125991 someano+  20   0    9352   3972   3296 R  12,5   0,1   0:00.02      
      1 root      20   0  181408   5804   3976 S   0,0   0,1   0:05.15      
      2 root      20   0       0      0      0 S   0,0   0,0   0:00.11      
      3 root       0 -20       0      0      0 I   0,0   0,0   0:00.00      
      4 root       0 -20       0      0      0 I   0,0   0,0   0:00.00      
      8 root       0 -20       0      0      0 I   0,0   0,0   0:00.00    
\end{lstlisting}
\subsubsection*{htop}
То же, но куда более красиво. Интерфейс простой и понятный, все удобно и круто, но требует установки
\begin{lstlisting}[language=bash]
  1  [                       0.0%]   5  [                       0.0%]       
  2  [||                     1.3%]   6  [                       0.0%]       
  3  [|                      0.7%]   7  [|                      0.7%]       
  4  [                       0.0%]   8  [|                      0.7%]       
  Mem[||||||||||||||||2.08G/3.73G]   Tasks: 92, 263 thr; 1 running          
  Swp[||               536M/8.00G]   Load average: 0.55 0.61 0.54           
                                     Uptime: 1 day, 08:58:42                
                                                                            
    PID USER      PRI  NI  VIRT   RES   SHR S CPU% MEM%   TIME+  Command    
 125857 someanoni  20   0  8332  4324  3412 R  0.7  0.1  0:00.09 htop       
  95156 someanoni  20   0 3273M  419M 41392 S  0.7 11.0 17:51.77 /usr/bin/t 
      1 root       20   0  177M  5804  3976 S  0.0  0.1  0:05.15 /sbin/init 
    345 root       20   0 78068   104     0 S  0.0  0.0  0:00.00 /usr/bin/l 
    357 root       20   0  8120  1228  1136 S  0.0  0.0  1:14.08 /usr/sbin/ 
\end{lstlisting}
\subsubsection*{sensors/powermetrix}
\subsubsection*{reboot}
Перезагрузка машины. Требует быть администратором при наличии других пользователей залогиненых на эту машину. Не имеет агрументов
\subsubsection*{shutdown}
Выключение. То же примечание что и для reboot. Принимает аргумент - время выключения.
\begin{lstlisting}[language=bash]
 $ shutdown now
\end{lstlisting}
\subsubsection*{date}
Показывает дату в удобном виде. Может и установить ее, но это требует администраторских полномочий

\begin{lstlisting}[language=bash]
 $ date
 Tue 24 MAR 2020 00:30:00
\end{lstlisting}

\subsubsection*{ping}
Проверка связи с хостом. Хост - адрес сайта или ip. Посылает тестовые пакеты пока не убьешь процесс. Ну или до количества переданного с ключом $-c$
\begin{lstlisting}[language=bash]
 $ ping -c 3 google.com
PING google.com (209.85.233.139) 56(84) bytes of data.
64 bytes from lr-in-f139.1e100.net (209.85.233.139): icmp_seq=1 ttl=46 time=22.8 ms
64 bytes from lr-in-f139.1e100.net (209.85.233.139): icmp_seq=2 ttl=46 time=24.3 ms
64 bytes from lr-in-f139.1e100.net (209.85.233.139): icmp_seq=3 ttl=46 time=22.8 ms

--- google.com ping statistics ---
3 packets transmitted, 3 received, 0% packet loss, time 3005ms
rtt min/avg/max/mdev = 22.776/23.192/24.285/0.633 ms
 $ ping 8.8.8.8
PING 8.8.8.8 (8.8.8.8) 56(84) bytes of data.
64 bytes from 8.8.8.8: icmp_seq=1 ttl=45 time=23.0 ms
64 bytes from 8.8.8.8: icmp_seq=2 ttl=45 time=22.9 ms
64 bytes from 8.8.8.8: icmp_seq=3 ttl=45 time=22.9 ms
64 bytes from 8.8.8.8: icmp_seq=4 ttl=45 time=22.9 ms
^C
--- 8.8.8.8 ping statistics ---
4 packets transmitted, 4 received, 0% packet loss, time 3003ms
rtt min/avg/max/mdev = 22.857/22.917/22.971/0.041 ms

\end{lstlisting}


\section*{Middle}
\section*{Pro}

\end{document}

